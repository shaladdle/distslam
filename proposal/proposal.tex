\documentclass[11pt]{article}

\usepackage[margin=1in]{geometry}
\usepackage{graphicx}

% Always place footnotes at the bottom of the page
\usepackage[bottom]{footmisc}

\title{
15-780 Final Project Proposal \\ 
Distributed Simultaneous Localization And Mapping \\
}
\author{Adam Wright, Nathan Slobody, Tim Kuehn}

\begin{document}

\maketitle

\section{Description}

\begin{itemize}
    \item Multiple robots map out the same environment, having no prior information about the environment

    \item Does having multiple robots do it increase the accuracy of the map?

    \item Does having multiple robots do it increase the speed at which you can map?

    \item How do we merge the maps? Veloso described an approach where multiple robots start in different locations and fuse their maps when they meet.
        \begin{itemize}
            \item One idea is that the robots don't merge their maps until they have seen two landmarks in common, so that they can orient themselves on the shared map.
        \end{itemize}

\end{itemize}

\section{Implementation}

\begin{itemize}
    \item Use iRobot Creates
        \begin{itemize}
            \item With to-be-determined level of sensors, communication, and onboard computing power
        \end{itemize}

    \item We can create some kind of maze and use their touch sensors, or use april tags and get some cheap cameras? My hesitation with using the touch sensors is how do we identify landmarks that we've hit when we hit them again? There's no way of touching a wall twice and saying ``that's the same chunk of wall''.. Or at least I don't get it.

    \item There is also the question of how the robots will communicate and share / store data.  We're not sure it can be done without onboard computers, though MV said that it could.  We're actually not even sure what equipment will come with the robots we're given yet.  

\end{itemize}

\section*{Project Plan}

\paragraph{75\%}
\begin{itemize}
    \item Ascertain CreBot capabilities: what sensors are included (bumb, rangefinder, vision, etc), what onboard computing power is available, and what storage and communication facilities can be used.
    \item Become familiar with existing literature and research on SLAM, probabilistic tools such as Kalman and particle filters, and distributed / cooperative robotics.
    \item Determine how landmarks will be represented: april tags, something more sophisticated with feature extraction by sensors, or something as simple as possible with bump sensors.
    \item Set up a testing location for the robots to map - either an artificial maze, a room in Gates Hillman Center, or another area.
    \item Learn or write any necessary control, navigation, communication or systems software for working with the robots.
    \item Write a working SLAM implementation for a single robot.
\end{itemize}

\paragraph{100\%}

\begin{itemize}
    \item Work out systems programming details of information sharing between two robots.
    \item Decide how two robots will merge their maps (e.g. when they have both seen two of the same landmarks).
    \item Get a potentially naive distributed SLAM implementation working for two robots.
\end{itemize}

\paragraph{125\%}
\begin{itemize}
    \item Iron out any wrinkles in the basic distributed SLAM implementation and get a more sophisticated/reliable/faster version working
    \item Possibly add an additional robot or two to ascertain the impact on speed and efficiency.
\end{itemize}

\begin{thebibliography}{12}
    \bibitem{thrun2005}
        S. Thrun, W. Burgard, and D. Fox, \emph{Probabilistic Robotics}, MIT Press, 2005.

    \bibitem{thrun2003}
        S. Thrun and Y. Liu, ``Multi-Robot SLAM with Sparse Extended Information Filters'', \emph{Proceedings of the 11th International Symposium of Robotics Research (ISRR'03)}, 2003.

    \bibitem{cunningham2010}
        A. Cunningham, M. Paluri, and F. Dellaert, ``DDF-SAM: Fully Distributed SLAM using Constrained Factor Graphs'', \emph{International Conference on Intelligent Robots and Systems (IROS)}, 2010.

\end{thebibliography}

\end{document}
