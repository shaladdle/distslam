\documentclass[11pt]{article}

\usepackage[margin=1in]{geometry}
\usepackage{graphicx}

% Always place footnotes at the bottom of the page
\usepackage[bottom]{footmisc}

\title{
15-780 Final Project Proposal \\ 
Distributed Simultaneous Localization And Mapping \\
}

\author{Adam Wright, Nathan Slobody, Tim Kuehn}

\begin{document}

\maketitle

\section{Introduction}

Robot perception is an important requirement for intelligent autonomous robots. For a robot to be able to use a search technique like $A*$ to find a path through an environment, or use some kind of planning to execute that path, the robot must first have a map of its environment. Some applications of robotics, such as search and rescue or indoor fire fighting do not have the luxury of having an accurate map beforehand. Simultaneous Localization And Mapping (SLAM) techniques, which build maps based only on sensor measurements, allow autonomous robots in these applications to succeed. Some applications, such as search and rescue, would benefit from using multiple robots to cover a wide area quickly. 

In this project, we will study robot perception by learning about and implementing multi-robot Simultaneous Localization And Mapping (SLAM). In particular we will focus on how robots share and combine their data and local maps to produce a single map of the environment all robots have explored thus far. Our goal is to implement a technique that does not add significant computing burden to the single-robot SLAM algorithm.

Professor Veloso has agreed to provide us with two robots from her lab. We will use some simple method of landmark identification such as touch sensors using an iRobot Create bumper 

We have the following logistics

We want to answer several questions

\begin{itemize}
    \item What should the starting conditions be? Should robots be allowed to know their initial locations?

    \item Does having multiple robots perform SLAM on slightly overlapping areas produce good quality maps in less time? How does this compare to a single robot?

    \item How much more accuracy can you get from multiple robots mapping out heavily overlapping areas at the same time?

    \item How do we merge the maps? Veloso described an approach where multiple robots start in different locations and fuse their maps when they meet.
        \begin{itemize}
            \item One idea is that the robots don't merge their maps until they have seen two landmarks in common, so that they can orient themselves on the shared map.
        \end{itemize}

\end{itemize}

\section{Implementation}

\begin{itemize}
    \item Use iRobot Creates
        \begin{itemize}
            \item With to-be-determined level of sensors, communication, and onboard computing power
        \end{itemize}

    \item We can create some kind of maze and use their touch sensors, or use april tags and get some cheap cameras? My hesitation with using the touch sensors is how do we identify landmarks that we've hit when we hit them again? There's no way of touching a wall twice and saying ``that's the same chunk of wall''.. Or at least I don't get it.

    \item There is also the question of how the robots will communicate and share / store data.  We're not sure it can be done without onboard computers, though MV said that it could.  We're actually not even sure what equipment will come with the robots we're given yet.  

\end{itemize}

\section*{Project Plan}

\paragraph{75\%}
\begin{itemize}
    \item Ascertain CreBot capabilities: what sensors are included (bump, rangefinder, vision, etc), what onboard computing power is available, and what storage and communication facilities can be used.
    \item Become familiar with existing literature and research on SLAM, probabilistic tools such as Kalman and particle filters, and distributed / cooperative robotics.
    \item Determine how landmarks will be represented: april tags, something more sophisticated with feature extraction by sensors, or something as simple as possible with bump sensors.
        \begin{itemize}
            \item If using april tags, become familiar with their identification vis-a-vis the available sensors
        \end{itemize}
    \item Set up a testing location for the robots to map - either an artificial maze, a room in Gates Hillman Center, or another area.
    \item Learn or write any necessary control, navigation, communication or systems software for working with the robots.
    \item Write a simple navigation algorithm for the robot to decide where to go / in what order to traverse the area.
    \item Write a working SLAM implementation for a single robot.
\end{itemize}

\paragraph{100\%}

\begin{itemize}
    \item Work out systems programming details of information sharing between two robots.
    \item Decide how two robots will merge their maps (e.g. when they have both seen two of the same landmarks, or when they find that they are next to each other somehow).
    \item Get a potentially naive distributed SLAM implementation working for two robots.
\end{itemize}

\paragraph{125\%}
\begin{itemize}
    \item Iron out any wrinkles in the basic distributed SLAM implementation and get a more sophisticated/reliable/faster version working
    \item Write a more efficient navigation algorithm for the robot to decide where to go / in what order to traverse the area.
    \item Possibly add an additional robot or two to ascertain the impact on speed and efficiency.
\end{itemize}

\begin{thebibliography}{12}
    \bibitem{thrun2005}
        S. Thrun, W. Burgard, and D. Fox, \emph{Probabilistic Robotics}, MIT Press, 2005.

    \bibitem{thrun2003}
        S. Thrun and Y. Liu, ``Multi-Robot SLAM with Sparse Extended Information Filters'', \emph{Proceedings of the 11th International Symposium of Robotics Research (ISRR'03)}, 2003.

    \bibitem{cunningham2010}
        A. Cunningham, M. Paluri, and F. Dellaert, ``DDF-SAM: Fully Distributed SLAM using Constrained Factor Graphs'', \emph{International Conference on Intelligent Robots and Systems (IROS)}, 2010.

\end{thebibliography}

\end{document}
